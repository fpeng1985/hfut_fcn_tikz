\documentclass[preview]{standalone}
\usepackage{ctex}
%graphics
\usepackage{xcolor}
\usepackage{tikz}
\usetikzlibrary{shapes.geometric, shapes.multipart, arrows, calc, through}
\usepackage[caption=false,font=footnotesize]{subfig}

\begin{document}
 

%\definecolor{nml-north-color}{HTML}{6666FF}
%\definecolor{nml-south-color}{HTML}{FF6666}
\definecolor{nml-north-color}{gray}{0.8}
\definecolor{nml-south-color}{gray}{0.2}
\begin{figure}[!t]
\centering


\subfloat[NML 元胞]{
\begin{tikzpicture}[scale=0.2]
\shade[draw=gray,thick,top color=nml-north-color,bottom color=nml-south-color,middle color=white] (-2.5,-5) rectangle (2.5,5);
\draw[->, line width=2pt] (0,-4.5) -- (0, 4.5);
\node at (0, -6) {逻辑`1'};

\shade[draw=gray,thick,top color=nml-south-color,bottom color=nml-north-color,middle color=white] (-11.5,-5) rectangle (-6.5,5);
\draw[->, line width=2pt] (-9,4.5) -- (-9, -4.5);
\node at (-9, -6) {逻辑`0'};

\shade[draw=gray,thick,left color=nml-south-color,right color=nml-north-color,middle color=white] (6.5,-5) rectangle (11.5,5);
\draw[->, line width=2pt] (7,0) -- (11, 0);
\node at (9, -6) {逻辑`null'};
\end{tikzpicture}}

\subfloat[纳米磁畴的耦合性]{%[Nano Magnet Coupling]{
\begin{tikzpicture}[scale=0.2]

\shade[draw=gray,thick,top color=nml-north-color,bottom color=nml-south-color,middle color=white] (-2.5,-5) rectangle (2.5,5);
\draw[->, line width=2pt] (0,-4.5) -- (0, 4.5);
\shade[draw=gray,thick,top color=nml-south-color,bottom color=nml-north-color,middle color=white] (-10.5,-5) rectangle (-5.5,5);
\draw[->, line width=2pt] (-8,4.5) -- (-8, -4.5);
\node at (-4, -6) {反铁磁性耦合}; %{Antiferromagnetic Coupling};

\shade[draw=gray,thick,left color=nml-south-color,right color=nml-north-color,middle color=white] (9,-2.5) rectangle (19,2.5);
\draw[->, line width=2pt] (9.5,0) -- (18.5, 0);
\shade[draw=gray,thick,left color=nml-south-color,right color=nml-north-color,middle color=white] (22,-2.5) rectangle (32,2.5);
\draw[->, line width=2pt] (22.5,0) -- (31.5, 0);
\node at (20.5, -6) {铁磁性耦合}; %{Ferromagnetic Coupling};

\end{tikzpicture}}

\end{figure}

\end{document}
