\documentclass[tikz]{standalone}


\usetikzlibrary{arrows.meta}
\colorlet{tile_color_1}{lightgray!20}
\colorlet{tile_color_2}{lightgray!50}
\colorlet{tile_color_3}{lightgray}
\colorlet{tile_color_4}{gray}
\def\sz{1cm}
\usetikzlibrary{math}
\tikzmath{
	%USE Clocking Scheme
	function use_val(\x,\y){
		return (mod(\y,2)!=0) ? ((mod(\y+1,4)!=0)?(1+mod(\x+1,4)):(1+mod(\x+3,4))):((mod(\y,4)==0)?(4-mod(\x+3,4)):(4-mod(\x+1,4)));
	};
	%
	function use(\row, \col) {
		\row = \row-1;
		\col = \col-1;
		int \i, \j, \c; 
		for \i in {0,...,\row}{
			for \j in {0,...,\col}{
				\c = use_val(\j, \i);
				{ \path (\j, \i) {node(\j\i)[fill=tile_color_\c]{}}; };
			};
		};
	};
	%TDD Clocking Scheme
	function tdd_val(\x,\y){
		return (mod(\y, 2)!=0) ? ((mod(\y+1, 4)!=0)?(1+mod(\x+1, 4)):(1+mod(\x+3, 4))):((mod(\y, 4)==0)?(1+mod(\x, 4)):(1+mod(\x+2, 4)));
	};
	%
	function tdd(\row, \col) {
		\row = \row-1;
		\col = \col-1;
		int \i, \j, \c; 
		for \i in {0,...,\row}{
			for \j in {0,...,\col}{
				\c = tdd_val(\j, \i);
				{ \path (\j, \i) {node(\j\i)[fill=tile_color_\c]{}}; };
			};
		};
	};
}

\begin{document}
\begin{tikzpicture}[
	x=\sz,
	y=\sz,
	minimum size=\sz,
	inner sep=0pt,
	outer sep=0pt,
	r/.style={->, >={Stealth[]},line width=1.4pt, black},
	v/.style={circle, draw, fill=white, line width=1pt, minimum size=\sz*.75, font=\large\boldmath},
	]
	

\begin{scope}
\tikzmath{use(4,6);}
\node[v](5)at(2,2){5};
\node[v](4)at(1,3){4};
\node[v](3)at(5,3){3};
\node[v](2)at(0,0){2};
\node[v](1)at(4,0){1};
\node[v](7)at(3,1){7};
\node[v](10)at(1,0){10};
\node[v](8)at(2,1){8};
\node[v](11)at(0,2){11};
\node[v](6)at(4,3){6};
\node[v](9)at(5,1){9};

\draw[r] (3) -- (5,2) -| (7.north east);
\draw[r] (4) -| (7.north west);
\draw[r] (5) -| (10);
\draw[r] (7.south east) |- (10.south east);
\draw[r] (2) |- (8.north west);
\draw[r] (7.south west) -- ++(0,-0.6) -| (8.south east);
\draw[r] (8.east) |- (11.north);
\draw[r] (10.south) -| (11.west);
\draw[r] (1.north west) -- (6.south west);
\draw[r] (3) -- ++(0,-1) -| (6.south east);
\draw[r] (6.north) -| (9.east);
\draw[r] (8) -- (9);

\end{scope}


%\begin{scope}[xshift=\sz*7]
%\tikzmath{tdd(6,6);}
%\node(5) [v] at (2,4) {5};
%\node(4) [v] at (2,3) {4};
%\node(3) [v] at (3,2) {3};
%\node(2) [v] at (4,2) {2};
%\node(1) [v] at (4,1) {1};
%\node(7) [v] at (3,3) {7};
%\node(10) [v] at (3,4) {10};
%\node(8) [v] at (4,3) {8};
%\node(11) [v] at (4,4) {11};
%\node(6) [v] at (5,2) {6};
%\node(9) [v] at (5,4) {9};

%\draw[r] (3) -- (7);
%\draw[r] (4) -- (7);
%\draw[r] (5) -- (10);
%\draw[r] (7) -- (10);
%\draw[r] (2) -- (8);
%\draw[r] (7) -- (8);
%\draw[r] (8) -- (11);
%\draw[r] (10) -- (11);
%\draw[r] (1) -| (6.south east);
%\draw[r] (3.south east) -- (6.south west);
%\draw[r] (6.north east) -- (9.south east);
%\draw[r] (8) -- (5,3) -- (9);

%\end{scope}
\end{tikzpicture}

\end{document}

