\documentclass[preview,border=0mm,convert={convertexe={magick},outext=.ps}]{standalone}
\usepackage[dvipdfm]{geometry}
%graphics
\usepackage{xcolor}
\usepackage{tikz}
\usetikzlibrary{shapes.geometric, shapes.multipart, arrows, calc, through, shadows}
\usepackage[caption=false,font=footnotesize]{subfig}


\begin{document}
\definecolor{trait-color}{RGB}{170,198,231}
\begin{figure}[!t]
\centering
\begin{tikzpicture}[
trait/.style={ rectangle, draw=black, rounded corners=2pt, fill=trait-color, drop shadow, text centered, anchor=north, text=black, text width=7cm, minimum width=3.5cm, minimum height=0.7cm},
myarrow/.style={->, >=open triangle 90, thick},
line/.style={-, thick},
node distance =2cm,
transform shape,
scale=0.45,
] 
\node (Clock)[trait, rectangle split, rectangle split parts=3]
    {
        \textbf{Clock}
        \nodepart{second} no data 
        \nodepart{third} {the common procedure of clock generation}
    };
    
\node (QCA)[trait, rectangle split, rectangle split parts=3]
    at ($(Clock.south)+(0,-0.6)$)
    {
        \textbf{QCA Clock}
        \nodepart{second} {QCA clock data}
        \nodepart{third}{the 4-phased clock generation of QCA}
    };    

\node (Energy)[trait, rectangle split, rectangle split parts=3]
    at ($(QCA.south)+(0,-0.6)$)
    {
        \textbf{QCADesigner-E's Clock}
        \nodepart{second}\small{no extra data}
        \nodepart{third}\small{the Gaussian shaped clock}
    
    };
    
\draw[myarrow] (QCA.north) -- (Clock.south);
\draw[myarrow] (Energy.north) -- (QCA.south);

\end{tikzpicture}
\end{figure}
\end{document}
