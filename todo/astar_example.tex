\documentclass[preview]{standalone}
\usepackage{tikz}
\usetikzlibrary{arrows.meta, math, calc}

\begin{document}
  \definecolor{cs1}{RGB}{89,89,91}
  \definecolor{cs2}{RGB}{129,130,132}
  \definecolor{cs3}{RGB}{210,210,212}
  \definecolor{cs4}{RGB}{168,169,173}
  \definecolor{inp}{RGB}{230,230,230}
  \definecolor{out}{RGB}{40,40,40}
  \definecolor{fix}{RGB}{0,0,0}
  
  \begin{tikzpicture}[
      x={(2cm, 0cm)},y={(0cm,-2cm)},
      c1/.style={shape=rectangle, fill=cs1, text=black, minimum size=2cm},
      c2/.style={shape=rectangle, fill=cs2, text=black, minimum size=2cm},
      c3/.style={shape=rectangle, fill=cs3, text=black, minimum size=2cm},
      c4/.style={shape=rectangle, fill=cs4, text=black, minimum size=2cm},
      ]
      \tikzmath{
        function use(\x,\y){
          return (mod(\y,2)!=0) ? 
          ((mod(\y+1,4)!=0)?(4-mod(\x,4)):(4-mod(\x+2,4))) :
          ((mod(\y,4)==0)?(1+mod(\x,4)):(1+mod(\x+2,4)));
        };
        %
        int \i, \j, \c; 
        for \i in {0,1,2,3}{
          for \j in {0,1,2,3}{
            \c = use(\i,\j);
            {
              \path 
                (\i,\j)                                node(\i\j)[c\c]{}
                ($(\i\j.center)!.7!(\i\j.south east)$) node{\large \c};
            };
          };
        };
      }
      
      \begin{scope}[
        every node/.style={shape=circle, draw=black, very thick, minimum size=1cm},
        ]
      \path 
        (1,1) node(1){1}
        (1,2) node(3){3}
        (2,2) node(2){2}
        ;
      \end{scope}

      \begin{scope}[
        ultra thick,
        -{Stealth[inset=0]},
        blue!50 
        ]
        
        \draw (1) -| ++(-1,-1) -| ++(3,3) -| (2);
        \draw  (1.south east) |- (2.north);
      \end{scope}
    
  \end{tikzpicture}
\end{document}

